\documentclass[10pt]{article}
\usepackage{graphics}                      % epsfig
\usepackage{epsfig}                        % epsfig

\topmargin=0pt
\headheight=0pt
\headsep=0pt
\oddsidemargin=0.0in
\textheight=9.25in
\textwidth=6.5in

\begin{document}

\begin{center}
{\bf TFOC documentation}
\end{center}

\section{Effective Medium Approximations}

In the sample structure, a mixed phase material material may be
specified using the format
\begin{eqnarray*}
 [ ~\lbrace\mbox{model}\rbrace~~f_1~~mat_A~~f_2~~mat_B~~\dots~]
\end{eqnarray*}
where the EMA (Effective Medium Approximation) model may be optionally specified
followed by volume fractions of various materials.  The sum $f_A+f_B+\dots$ need not be
unity as TFOC will normalize the values. Each of the materials mat$_i$ may either be
a specific material from the database or may itself be a mixed phase model.  

Some specific examples are
\begin{itemize}
\item\enskip [ 0.8 Si 0.2 SiO2 ]  30
\item\enskip [ PARALLEL 0.3 Si 0.4 Si3N4 0.3 SiO2 ] 120
\item\enskip [ 0.8 Si 0.1 [ 0.3 SiO2 0.7 Si3N4 ] 0.1 Al2O3 ] 80
\item\enskip [ LOOYENGA 0.8 Si 0.1 [ BRUGGEMAN 0.3 SiO2 0.7 Si3N4 ] 0.1 Al2O3 ] 250
\end{itemize}

The code implements sevearl EMA models which are strictly valid in specific
sample constructs.
\begin{itemize}
\item {\bf Series (default):} The series mixing is a simple weighted sum of the
dielectric constants of the elements.  This model is symmetric and valid for any
number of components.
\begin{eqnarray*}
\epsilon_{\rm eff} &=& \sum f_i \epsilon_i
\end{eqnarray*}
\item {\bf Parallel:} The parallel model is the other limit for the dielectric
constant of an arbitrary mixture.  Like the series model, it is valid for an arbitrary
number of components and is symmetric in all components:
\begin{eqnarray*}
\frac{1}{\epsilon_{\rm eff}} &=& \sum \frac{f_i}{\epsilon_i}
\end{eqnarray*}
\item {\bf Looyenga:} This model uses a weighted sum of the dielectric constant to the
1/3 power.  See H. Looyenga, {\it Dielectric constants of heterogeneous mixtures}, Physica {\bf 31}, 401-406 (1965).
Again, like the previous two, it has the advantage of being extensible to multiple
components and all phases are treated symmetrically.
\item {\bf Bruggeman:} Bruggeman in the 1930's created the master of all EMAs.  The
form is deceptively simple and symmetric in any number of components, but as it does
not have a closed form solution it is implemented for only two components in the code.
The model is based on the equation:
\begin{eqnarray*}
\sum f_i \frac{\epsilon_i-\epsilon_{\rm eff}}{\epsilon_i+(n-1)\epsilon_{\rm eff}} = 0
\end{eqnarray*}
where $n$ is the dimensionality of the mixture (3 for normal 3D).  For two terms, $\epsilon_{\rm eff}$
can be written as
\begin{eqnarray*}
\epsilon_{\rm eff} &=& \frac{\epsilon_A(3f_A-1)+\epsilon_B(2-3f_A) + \sqrt{[\epsilon_A(3f_A-1)+\epsilon_B(2-3f_A)]^2+8\epsilon_A\epsilon_B}}{4}
\end{eqnarray*}
\item {\bf Maxwell-Garnett:} This EMA was derived to explain the color in glasses and is valid primarily
in the low volume fraction limit where the inclusions do not interact.  The lower volume fraction component
is taken as the inclusion ($\epsilon_i)$ while the majority component is the matrix ($\epsilon_m)$.  This model is
implemented for only two components and is not symmetric in the two elements.  The equation takes the form:
\begin{eqnarray*}
\frac{\epsilon_{\rm eff}-\epsilon_m}{\epsilon_{\rm eff}+2\epsilon_m} &=& f_i \left ( \frac{\epsilon_i-\epsilon_m}{\epsilon_i+2\epsilon_m} \right ) \\
\epsilon_{\rm eff} &=& \left [ \frac{\epsilon_i(1+2f_i) - \epsilon_m(2f_i-2)}{\epsilon_m(2+f_i)+\epsilon_i(1-f_i)} \right ]\epsilon_m
\end{eqnarray*}
\end{itemize}

\par The models are derived for pure dielectrics.  TFOC blindly extends these to the complex dielectric constant
and uses $n = \sqrt{\epsilon}$.  There is discussion of limits for complex dielectric constants which may later
be implemented:

\vskip 1truein

\noindent{\bf References:}
\begin{enumerate}
\item D.A. Bruggeman, Annalen der Physik, {\bf 24} (1933), 636-664.
\item O. Wiener, Abh. D. Kongl. Sachs. Ges. d Wiss, math.-phys. K1, {\bf 32} (1912), 509.
\item D.J Bergman, Phys. Rev. Lett. {\bf 44} (1980), 1285-1287.
\item G.W. Milton, Appl. Phys. Lett. {\bf 37} (1980), 300-302.
\end{enumerate}

\end{document}
\end{article}
\bye
